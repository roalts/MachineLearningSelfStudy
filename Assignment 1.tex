\documentclass[a4paper, 12pt]{article}

\usepackage[english]{babel}
\usepackage[utf8]{inputenc}
\usepackage{amsmath}
\usepackage{graphicx}
\usepackage[colorinlistoftodos]{todonotes}

\title{Assignment - 1}

\author{Raghav Apoorv}

\date{}

\begin{document}
\maketitle

\section{Why study Machine Learning?}

Machine Learning is study which shows that Machine themselves can learn from Data. It is a branch that has grown out of the works of Artificial Intelligence. It seeks new capabilities of what a computer can do. It is a field of study that gives computers the ability to learn without being explicitly programmed for each event or incident.\\

A computer program is said to learn from experience, with respect to some task and some performance measure, if its performance to perform that task improves with experience, then it is a well posed Learning Problem. Things like Database Mining, which have grown from the growth of automation and web. Google has implemented highly efficient Machine Learning algorithms, using which they collect each and every information about their user, as to what all they are doing online. All the recommendations made by google, spam filters, search engines, optical character recognition or even the ads on your facebook profile which learn about the sites you visit and log in using facebook account, and on the basis of learning it displays those ads, all these involves Machine Learning Algorithms.\\

When a Machine Learns itself out of it?s own experience that is Machine Learning. For example, if a developer develops a game, every time the ma- chine plays the game, it stores the board positions and moves, and keep on learning from each game played by it and gaining experience, so that the next game the machine plays it will keep in mind all the previous games.\\

Machine Learning is very useful when Big Data comes in, pattern recognition, it helps to find, figure out useful patterns in massive sets of data which can?t be done by a human or a explicitly programmed computer software. Suppose your email program watches which emails you do or do not mark as spam, and based on that learns how to better filter spam. So your email learns to classify which mail as spam or not.

\section{What are the different types of Problems?}

Machining Learning problems are basically divided on what types of learning feedback is available for the system through which it can learn. 

The problems can be divided into three broad categories: \dots

\begin{enumerate}
\item Supervised Learning
\item Unsupervised Learning
\item Reinforcement Learning
\end{enumerate}

\subsection{Supervised Learning}

In Supervised Learning the \'right answers\' are given. It is the machine learning task in which the function or algorithm has to inferred from labeled data set or training set, that is the information about the data set is provided and can be plotted on the graph with proper y-axis and x-axis.\\

The training data consist of a set of training examples. In supervised learning, each example is a pair consisting of an \"input\" variable or feature,(x\'s) typically a vector and a desired \"output\" variable or \"target\" feature (y\'s) also called the supervisory signal and \"m\" is the number of training examples.\\
A supervised learning algorithm analyzes the training data and produces an inferred function or algorithm, which can be used for mapping new or similar examples. The most optimised algorithm will correctly determine the class of the labels for unseen instances, so that the learning algorithm can generalize the situation from the training data set and apply them to the unseen situations in a meaningful manner.


\end{document}